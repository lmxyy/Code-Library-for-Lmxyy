\documentclass[11pt]{article}
\usepackage{xeCJK}
\usepackage{amsmath}
\usepackage{amsthm}
\usepackage{enumerate}

\begin{document}
	\section{有上下界网络流}
	\begin{enumerate}[1.]
		\item \textbf{无源汇有上下界可行流}
			
			设原来源点为$Source$,汇点是$Sink$。新建一个超级源$SuperSource$和超级汇$SuperSink$。对于原网络中的每一条边$u \rightarrow v$,上界$U$,下界$L$,将它拆分为三条边:
			\begin{enumerate}[(1)]
				\item $u \rightarrow SuperSink$,容量为$L$。
				\item $SuperSource \rightarrow v$,容量为$L$。
				\item $u \rightarrow v$,容量为$U-L$。
			\end{enumerate}
			最后添加边$Sink \rightarrow Source$,容量为$+\infty$。在新建的网络上,计算从$SuperSource$到$SuperSink$的最大流。若每条从$SuperSource$发出的边都满流,说明存在可行流,否则不。每条边实际流量为容量下界$+$附加流中它的流量。
		\item \textbf{有源汇有上下界最大流}
		
			在``无源汇有上下界可行流''建图上,新增一条$Sink \rightarrow Source$的边,容量为$+\infty$。先判断是否存在可行流,若存在可行流,接着在原图中$Source \rightarrow Sink$最大流增广加上原可行流即为最大流答案。(若存在可行流,去掉下界后最大流即为原图有源汇有上下界最大流)
		\item \textbf{有源汇有上下界最小流}
			
	\end{enumerate}
\end{document}