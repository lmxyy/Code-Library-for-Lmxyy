
%% \documentclass[a4paper]{article}
%% %%\documentclass[11pt]{article}
%% \usepackage{graphicx}
%% \usepackage{longtable}
%% \usepackage{latexsym}
%% \usepackage{multirow}
%% \usepackage{amssymb}
%% \usepackage{xeCJK}
%% \usepackage{tabularx}
%% \usepackage{algorithm}
%% \usepackage{algorithmic}
%% \usepackage{supertabular}
%% \usepackage{multirow}
%% \usepackage{indentfirst}
%% \usepackage{xcolor}
%% \usepackage{listings}
%% \usepackage{courier}
%% \usepackage{geometry}
%% \usepackage{fancyhdr}
%% \usepackage{fancyvrb}
%% \usepackage{fancybox} 
%% \usepackage{amsmath,amsfonts,amssymb,graphicx}
%% \usepackage{subfigure} 
%% \usepackage{bm}
%% \usepackage{multicol}
%% \usepackage{abstract}
%% \usepackage{mathrsfs}
%% \usepackage[cache=false]{minted}
%% \newcommand{\tabincell}[2]{\begin{tabular}{@{}#1@{}}#2\end{tabular}}
%% \setlength{\parindent}{2em}
%% \tolerance=1000
%% \lstset{language=C++}%这条命令可以让LaTeX排版时将C++键字突出显示
%% \lstset{breaklines}%这条命令可以让LaTeX自动将长的代码行换行排版
%% \lstset{extendedchars=false}%这一条命令可以解决代码跨页时,章节\题,页眉等汉字不显示的问题
%% \lstset{
%%          basicstyle=\scriptsize\ttfamily{}, % Standardschrift
%%          numbers=left,               % Ort der Zeilennummern
%%          numberstyle=\tiny,          % Stil der Zeilennummern
%%          %stepnumber=5,               % Abstand zwischen den Zeilennummern
%%          numbersep=10pt,              % Abstand der Nummern zum Text
%%          tabsize=4,                  % Groesse von Tabs
%%          %extendedchars=true,         %
%%          breaklines=true,            % Zeilen werden Umgebrochen
%%          keywordstyle=\color{red},
%%          stringstyle=\color{white}\ttfamily, % Farbe der String
%%          showspaces=false,           % Leerzeichen anzeigen ?
%%          showtabs=false,             % Tabs anzeigen ?
%%          xleftmargin=17pt,
%%          framexleftmargin=17pt,
%%          framexrightmargin=5pt,
%%          framexbottommargin=0pt,
%%          backgroundcolor=\color{lightgray},
%%          showstringspaces=false      % Leerzeichen in Strings anzeigen ?
%% }          
%% \providecommand{\alert}[1]{\textbf{#1}}
           
%% %代码高亮  
%% \geometry{margin=1in}
           
%% %字体设置  
%% %\setmainfont{PingFangSC-Light}
%% %\setCJKmonofont{PingFangSC-Light}
%% %\setCJKmainfont[BoldFont={PingFangSC-Regular}]{PingFangSC-Light}
          
\documentclass[a4paper]{article} 
\usepackage{bm}
\usepackage{cmap}
\usepackage{ctex}
\usepackage{cite}
\usepackage{color}
\usepackage{float}
\usepackage{xeCJK}
\usepackage{amsthm}
\usepackage{amsmath}
\usepackage{amssymb}
\usepackage{setspace}
\usepackage{geometry}
\usepackage{hyperref}
\usepackage{enumerate}
\usepackage{indentfirst}
\usepackage[cache=false]{minted}

%代码高亮
\geometry{margin=1in}

%字体设置
%\setmainfont{PingFangSC-Light}
%\setCJKmonofont{PingFangSC-Light}
%\setCJKmainfont[BoldFont={PingFangSC-Regular}]{PingFangSC-Light}


\newcommand{\cppcode}[1]{
    \inputminted[mathescape,
    			tabsize=4
    			]{cpp}{source/#1}
}

\newcommand{\javacode}[1]{
    \inputminted[mathescape,
    			tabsize=4
    			]{java}{source/#1}
}
        
\title{Code Library For lmxyy}
\author{lmxyy}
\date{\today}
\begin{document}
\maketitle
\newpage
\tableofcontents
\newpage
\section{计算几何}
	\subsection{半平面交}
		\cppcode{计算几何/半平面交.cpp}

	\subsection{常用公式}
		\cppcode{计算几何/常用公式.cpp}

	\subsection{凸包}
		\cppcode{计算几何/凸包.cpp}

	\subsection{旋转卡壳}
		\cppcode{计算几何/旋转卡壳.cpp}
\section{数据结构}
	\subsection{KDtree}
		\cppcode{数据结构/KDtree.cpp}
	\subsection{lct}
		\cppcode{数据结构/lct.cpp}

	\subsection{splay}
		\cppcode{数据结构/splay.cpp}

	\subsection{树链剖分}
		\cppcode{数据结构/树链剖分.cpp}

	\subsection{主席树}
		\cppcode{数据结构/主席树.cpp}
	\subsection{左偏树}
		\cppcode{数据结构/左偏树.cpp}
\section{数论算法}
	\subsection{bsgs}
		\cppcode{数论算法/bsgs.cpp}
	\subsection{NTT}
		\cppcode{数论算法/NTT.cpp}
	\subsection{pollard rho}
		\cppcode{数论算法/pollard rho.cpp}
	\subsection{扩展欧几里德}
		\cppcode{数论算法/扩展欧几里德.cpp}
	\subsection{线筛}
		\cppcode{数论算法/线筛.cpp}
\section{数值算法}
	\subsection{fft}
		\cppcode{数值算法/fft.cpp}
\section{图论算法}
	\subsection{dinic}
		\cppcode{图论算法/dinic.cpp}
	\subsection{Hungry}
		\cppcode{图论算法/Hungry.cpp}
	\subsection{isap}
		\cppcode{图论算法/isap.cpp}
	\subsection{tarjan}
		\cppcode{图论算法/tarjan.cpp}
	\subsection{zkw}
		\cppcode{图论算法/zkw.cpp}
	\subsection{树分治}
		\cppcode{图论算法/树分治.cpp}
	\subsection{朱刘算法}
		\cppcode{图论算法/朱刘算法.cpp}
	\subsection{最小费用最大流spfa}
		\cppcode{图论算法/最小费用最大流spfa.cpp}
\section{字符串}
	\subsection{ac自动机}
		\cppcode{字符串/ac自动机.cpp}
	\subsection{manacher}
		\cppcode{字符串/manacher.cpp}
	\subsection{后缀数组}
		\cppcode{字符串/后缀数组.cpp}
	\subsection{后缀自动机}
		\cppcode{字符串/后缀自动机.cpp}
	\subsection{回文自动机}
		\cppcode{字符串/回文自动机.cpp}

\end{document}
